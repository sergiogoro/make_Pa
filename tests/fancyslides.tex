


\documentclass{fancyslides} 
\usepackage[utf8]{inputenc}

% per ser processat amb xelatex
\usepackage{fontspec}
\fontspec[Ligatures=TeX]{Lato}

%%% Beamer settings (do not change)
\usetheme{default} 
\setbeamertemplate{navigation symbols}{} %no navigation symbols
\setbeamercolor{structure}{fg=\yourowntexcol} 
\setbeamercolor{normal text}{fg=\yourowntexcol} 

%%%%%%%%%%%%%%%%%%%%%%%%%
%%% CUSTOMISATIONS %%%%%%
%%%%%%%%%%%%%%%%%%%%%%%%%

% THE FOLLOWING COLOURS ARE PREDEFINED IN THE CLASS
%bi -- WHITE
%cz -- BLACK
%sz -- GRAY
%nieb -- BLUE
%ziel -- GREEN
%pom -- ORANGE
%% YOU CAN DEFINE YOUR OWN COLOUR TO USE HERE. SEE MAN.PDF


%%%% SLIDE ELEMENTS
\newcommand{\structureopacity}{0.75} %opacity for the structure elements (boxes and dots)
\newcommand{\strcolor}{ziel} %elements colour (predefined nieb; pom; ziel)

%%%% TEXT COLOUR
\newcommand{\yourowntexcol}{cz}



%%%%%%%%%%%%%%%%%%%%%%%%%
%%% dades pel Títol %%%%
%%%%%%%%%%%%%%%%%%%%%%%%%
\newcommand{\titlephrase}{Prova de la classe fancyslides}
\newcommand{\name}{Joan Queralt Gil}
\newcommand{\affil}{cata\LaTeX{}}
\newcommand{\email}{jqueralt @ gmail.com}
%%%%% comandaments introduïts:
\newcommand{\event}{\color{DarkGreen}{Esdeveniment i Any}} %comandament propi per indicar l'esdeveniment on es presenta
\newcommand{\opacity}{0.75}          			   %comandament propi per variar l'opacitat de les imatges de fons    
\newcommand{\sourcestext}{Fonts}
\newcommand{\thankyoutext}{Gr\`acies}

%%%%%%%%%%%%%%%%%%%%%%%%%%%%%%%%%%%%%%%%%%%%%%%%%%%%%%%%%%%%%%%%
\begin{document}

\startingslide % genera la diapositiva del títol amb les dades donades

%% imatges: http://www.flickr.com/photos/perpetualplum/
\fbckg{robin} %.jpg la imatge de fons

\begin{frame}
\pointedsl{Pitroig americ\`a} %diapositiva amb text dins un cercle
\end{frame}

\begin{frame}
%diapositiva sense text per mostrar el fons
\end{frame}


\begin{frame}
\framedsl{El Pitroig americ\`a:\ Turdus migratorius)} %diapositiva amb text dins un rectangle
\end{frame}

\begin{frame} %diapositiva amb l'entorn \itemize estàndard
\itemized{
\item ocell cantaire migratori
\item de la fam\'ilia del Tord
\item se'l troba de M\`exic al Canad\`a
}
\end{frame}

\fbckg{blank}   %canvi d'imatge de fons

\begin{frame}   %diapositiva per posar les fonts documentals
\sources{
\includegraphics[scale=0.2]{robin} \ \href{http://www.flickr.com/photos/41304517@N00/6803496340}{\textit{Sure sign of Spring}} \\
Imatge: Flickr CC de \href{http://www.flickr.com/photos/blmiers2/}{blmiers2}\\
Text: \href{http://en.wikipedia.org/wiki/American_Robin}{\textbf{American Robin} from Wikipedia, the free encyclopedia}
}
\end{frame}

\begin{frame}
  \thankyou   %%%% última diapositiva per donar les gràcies
\end{frame}

\end{document}