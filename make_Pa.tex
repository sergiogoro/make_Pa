%%%%%%%%%%%%%%%%%%%%%%%%%%%%%%%%%%%%%%%%%
% Fancyslides Presentation
% LaTeX Template
% Version 1.0 (30/6/13)
%
% This template has been downloaded from:
% http://www.LaTeXTemplates.com
%
% The Fancyslides class was created by:
% Paweł Łupkowski (pawel.lupkowski@gmail.com)
%
% License:
% CC BY-NC-SA 3.0 (http://creativecommons.org/licenses/by-nc-sa/3.0/)
%
%%%%%%%%%%%%%%%%%%%%%%%%%%%%%%%%%%%%%%%%%

%----------------------------------------------------------------------------------------
% PACKAGES AND OTHER DOCUMENT CONFIGURATIONS
%----------------------------------------------------------------------------------------

\documentclass{fancyslides}

\usepackage[utf8]{inputenc} % Allows the usage of non-english characters
\usepackage{times} % Use the Times font
\usepackage{booktabs} % Allows the use of \toprule, \midrule and \bottomrule in tables
\graphicspath{{images/}} % Location of the slide background and figure files

% Beamer options - do not change
\usetheme{default} 
\setbeamertemplate{navigation symbols}{} % Disable the slide navigation buttons on the bottom of each slide
\setbeamercolor{structure}{fg=\yourowntexcol} % Define the color of titles and fixed text elements (e.g. bullet points)
\setbeamercolor{normal text}{fg=\yourowntexcol} % Define the color of text in the presentation

%------------------------------------------------
% COLORS
% The following colors are predefined in this class: white, black, gray, blue, green and orange

% Define your own color as follows:
%\definecolor{pink}{rgb}{156,0,151}

\newcommand{\structureopacity}{0.75} % Opacity (transparency) for the structure elements (boxes and circles)

\newcommand{\strcolor}{blue} % Set the color of structure elements (boxes and circles)
\newcommand{\yourowntexcol}{white} % Set the text color

%----------------------------------------------------------------------------------------
% TITLE SLIDE
%----------------------------------------------------------------------------------------

%\newcommand{\titlephrase}{Make Pa} % Presentation title
%\newcommand{\name}{Sergio} % Presenter's name
%\newcommand{\affil}{} % Presenter's institution
%\newcommand{\email}{sergiogoro86@gmail.com} % Presenter's email address

\begin{document}

%\startingslide % This command inserts the title slide as the first slide

%----------------------------------------------------------------------------------------
%----------------------------------------------------------------------------------------
% PRESENTATION SLIDES
%----------------------------------------------------------------------------------------
%----------------------------------------------------------------------------------------

%------------------------------------------------
% Pointed
%------------------------------------------------

\fbckg{1.jpg} % Slide background image
\begin{frame}
\pointedsl{\$ Make Pa \ldots} % Text in this environment is printed in a circle and will be made large and uppercase - if you need to fit more text in you can reduce the font size within the \pointedsl{} bracket as usual, e.g. \pointedsl{\large smaller main point}
\end{frame}

%------------------------------------------------
% Framed
%------------------------------------------------

%\fbckg{2.jpg} % Slide background image
%\begin{frame}
%\framedsl{Qu\`e es el Pa?} % Text in this environment will be made large, uppercase and will wrap multiple lines
%\end{frame}

%------------------------------------------------
% Itemized
%------------------------------------------------

\fbckg{2.jpg} % Slide background image
\begin{frame}
\itemized{ % This environment simply prints a series of bullet points
\item Pa amb llevat
\item Massa madre
\item Pa de massa madre
}
\end{frame}

%------------------------------------------------
% \pitem
%------------------------------------------------

%\fbckg{2.jpg} % Slide background image
%\begin{frame}
%\framedsl{\pitem{Massa madre} \pitem{Pa amb massa madre} \fitem{Pa amb llevat ¿quimic?}} % Text in \pitem commands will be printed one after another on separate slides until all are displayed
%\end{frame}

%------------------------------------------------
% Enumerate
%------------------------------------------------

\fbckg{2.jpg} % Slide background image
\begin{frame}
\misc{ % Anything can be placed inside the \misc{} command
\Huge
\$ \texttt{apropos Pa}\\
Pa amb llevat\\
Massa madre\\
Pa de massa madre\\
%\begin{enumerate}
%\centering
%\item Pa amb llevat
%\item Massa madre
%\item Pa de massa madre
%\end{enumerate}
}
\end{frame}

%------------------------------------------------
% Table
%------------------------------------------------

\fbckg{2.jpg} % Slide background image
\begin{frame}
\misc{ % Anything can be placed inside the \misc{} command
%Tables can be included with the \texttt{\textbackslash misc\{\}} command:
Trade-off
\begin{table}[h]
\begin{tabular}{l l l}
\toprule
\textbf{Pa amb \ldots} & \textbf{Llevat} & \textbf{Massa madre}\\
\midrule
time & Rapid & Lento \\
top & Senzill & +Elaborat \\
stdout & Bonn\`isim & \textsc{m.b.h} \\
\bottomrule
\end{tabular}
\end{table}
}
\end{frame}

%------------------------------------------------
% Image slide
%------------------------------------------------

\fbckg{2.jpg} % Slide background image
\begin{frame}
\misc{ % Anything can be placed inside the \misc{} command
Figures can also be included with the \texttt{\textbackslash misc\{\}} command:
\begin{figure}[h]
\includegraphics[width=0.4\linewidth]{placeholder}
\end{figure}
}
\end{frame}

%------------------------------------------------
% Thank you
%------------------------------------------------

\fbckg{mans_a_la_massa.jpg} % Slide background image
\begin{frame}
\thankyou % Inserts a thank you slide
\end{frame}

%------------------------------------------------
% Credits
%------------------------------------------------

\fbckg{blank} % A blank background can be used instead of an image
\begin{frame}
\sources{ % An environment for giving credit for slide backgrounds, images will need to be scaled down if there are more than two
\includegraphics[scale=0.048]{1.jpg} \ flickr/lovelornpoets\\
\includegraphics[scale=0.048]{1.jpg} \ youtube/ibanyarza\\
}
\end{frame}

%------------------------------------------------
% Resources
%------------------------------------------------

\fbckg{2.jpg} % Slide background image
\begin{frame}
\misc{ % Anything can be placed inside the \misc{} command
Resources:
\begin{figure}[h]
\includegraphics[width=0.4\linewidth]{Iban_pancasero}
\end{figure}
elforodelpan.com\\
lamemoriadelpan.com
}
\end{frame}

%----------------------------------------------------------------------------------------

\end{document}
